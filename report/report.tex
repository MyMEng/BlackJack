% Report --- ks1591
\documentclass[12pt,a4paper,twocolumn]{article}
\usepackage{times} % times font
\usepackage{mathptmx} % times font in maths
\usepackage{fullpage}
\usepackage[top=0.7in, bottom=0.7in, left=0.7in, right=0.7in]{geometry}
\usepackage{multirow} %in tables
\usepackage{caption} % in tables
\pagenumbering{gobble}
\newcommand{\HRule}{\rule{\linewidth}{0.5mm}}

\usepackage[pdftex]{graphicx}
\usepackage{lipsum}
\usepackage{amsmath}

% for multi figures
\usepackage{graphicx}
\usepackage{caption}
\usepackage{subcaption}
\usepackage{float}

\usepackage[usenames,dvipsnames]{color}

\usepackage[toc,page]{appendix}

\begin{document}

\twocolumn[
\begin{@twocolumnfalse}
\begin{center}
	\begin{large}
	{\HRule \\[0.2cm]}
	\textsc{Artificial intelligence algorithms for BlackJack}
	{\HRule \\[0.3cm]}
	\end{large}

	\begin{minipage}{ 0.44\textwidth }
		\begin{flushleft}
			Kacper \textbf{Sokol}\\
			\texttt{ks1591} --- 3GGK1
		\end{flushleft}
	\end{minipage}
	\begin{minipage}{ 0.44\textwidth }
		\begin{flushright}
			{COMS30106 $|$ Artificial Intelligence \&\\Logic Programming\\
			Assignment 2: BlackJack\\[0.3cm]}
		\end{flushright}
	\end{minipage}
\end{center}
\end{@twocolumnfalse}
] % \lipsum[1]~\\[0.4cm]

\begin{abstract}
This paper reviews most common strategies used by \emph{BlackJack}\footnote{For more details see:\\ http://en.wikipedia.org/wiki/Blackjack ~.} players to decrease \emph{house edge}\footnote{A statistical advantage of the house(casino) over player.}. Algorithms such as: \emph{deterministic table}, \emph{deck probabilities}, \emph{stand on all} and combination of first two are tested.\\
For this purpose BlackJack engine is implemented in \texttt{SWI-Prolog} programming language and \texttt{R} statistical package, where the later is accessed via \texttt{R..eal}\footnote{For more details see:\\ http://www.swi-prolog.org/pack/list?p=real ~.} interface.\\
Finally, results of the experiment are presented and discussed.\\

\begin{small}
\noindent
The source code together with this report is available at:\\ https://github.com/So-Cool/BlackJack ~.\\
The website also contains short, step-by-step description how to run the program.
\end{small}
\end{abstract}

\section*{\texttt{Rules}}
\subsection*{Game rules}
To begin with, we describe point counting. While calculating score of a hand we count: \emph{number cards} with their numerical value; \emph{face cards}(jack, queen, king) are treated as $10$ points and \emph{ace} is counted as $1$ or $11$, with value chosen by a player.\\
Furthermore, we distinguish two types of a hand: \emph{soft} and \emph{hard}. The first one contains an \emph{ace} and the later one does not.\\
Finally, during each round a player can either: \emph{hit} i.e.\ take a card from the deck or \emph{stand} i.e.\ refuse to take a card.\\

To sum up, each person plays \emph{only} against the dealer and the goal of each player is to either get $21$ points or to score more than the dealer simultaneously not exceeding $21$ points.\\

\subsection*{The house}
There are two house strategies implemented; both are documented to be the most popular among casinos worldwide, namely: \emph{H17} and \emph{S17}.\\
Both these strategies indicate to \emph{hit}, if current score is less than $17$ and \emph{stand}, if the score is more than $17$. The difference is that \emph{H17} dictates dealer to \emph{hit} on \emph{soft} $17$ and stand on hard $17$, while \emph{S17} defines to \emph{stand} on all types of $17$. In general, \emph{S17} is regarded as more beneficial for players.\\

Moreover, the advantage of house is also built upon the fact that only one card of the dealer is \emph{face up} i.e.\ only one card is revealed to the players. Also the house plays as a last in each round therefore being aware of the table state.\\


\section*{\texttt{The experiment}}
\subsection*{Parameters}
To examine aforementioned strategies a BlackJack game is implemented with a number of parameters being user configurable:
\begin{description}
\item{\texttt{Plays}} \hfill\\
Number of games to be played.
\item{\texttt{Decks}} \hfill\\
Number of decks used in a single game.
\item{\texttt{Players}} \hfill\\
Number of AI players. AI1 plays \emph{deterministic table}, AI2 plays \emph{deck probabilities}, AI3 plays \emph{probabilistic choice}, AI4 plays \emph{stand on all}, and all other defined players play \emph{deterministic table} if not defined otherwise.
\item{\texttt{PlayerMode}} \hfill\\
The gameplay can be either defined to be: \emph{experimental}---no user interaction required, this mode is mainly used to test AI algorithms; or \emph{interactive}---user can play along AI algorithms.
\item{\texttt{ShuffleMode}} \hfill\\
The simulation performs \emph{riffle shuffle}(see below); it can be either \emph{deterministic}---there is only one (fair)coin toss to decide whether left or right pile starts the shuffle; or \emph{random}---coin is tossed for each card(from left pile) to decide whether it goes on the top or on the bottom(of the card from right pile).
\item{\texttt{InitShuffles}} \hfill\\
Number of deck shuffles before the game starts.
\item{\texttt{Shuffles}} \hfill\\
Number of deck shuffles in between games.
\item{\texttt{Dealer}} \hfill\\
Dealer strategy to be used: \emph{H17} and \emph{S17}(as described above).
\end{description}

\subsection*{Shuffling}
The simulation performs \emph{riffle shuffle}. .\\


\section*{\texttt{Strategies}}
The following strategies were employed:
\begin{description}
\item{\textbf{Deterministic table}} \hfill\\
The action taken by AI module was predetermined according to widely used BlackJack table(Table~\ref{tab:det}). The table was simplified in a way that \emph{split} is not allowed i.e\ when the player is dealt with two same valued cards he cannot split his hand into two separate hands each containing one card.
\item{\textbf{Deck probabilities}} \hfill\\
In the experiment the decks are shuffled using \emph{riffle shuffle} i.e.\ the deck is split into halves and then it is combined by interchangeably dropping cards from each half. Aforementioned fact together with observable number of shuffles lead to probability estimates of cards being at given position in the deck, which are used to estimate risk of hold and hit. This yields the game mechanism which always takes the least risky action.
\item{\textbf{Probabilistic choice}} \hfill\\
It combines \emph{deterministic table} and \emph{deck probabilities} by creating a probability sampling vector over all cards that can be dealt next for the player and dealer to extend action determined by the tables.
\item{\textbf{Stand on all}} \hfill\\
This strategy is used to create a baseline to compare and evaluate all implemented algorithms. The strategy simply never takes additional cards.
\end{description}

\begin{table}[htop]
  \tiny
  \begin{tabular}{|l|c|c|c|c|c|c|c|c|c|c|}
    \hline
    \multicolumn{1}{ |l| }{\multirow{3}{*}{Player hand} } & \multicolumn{10}{ |c| }{Dealer's face-up card}\\
    \cline{2-11}
     & \multicolumn{10}{ |c| }{Hard totals}\\
    \cline{2-11}
     & 2 & 3 & 4 & 5 & 6 & 7 & 8 & 9 & 10 & A \\
    \hline
    17--20 & \colorbox{red}{S} & \colorbox{red}{S} & \colorbox{red}{S} & \colorbox{red}{S} & \colorbox{red}{S} & \colorbox{red}{S} & \colorbox{red}{S} & \colorbox{red}{S} & \colorbox{red}{S} & \colorbox{red}{S} \\
    \hline
    16 & \colorbox{red}{S} & \colorbox{red}{S} & \colorbox{red}{S} & \colorbox{red}{S} & \colorbox{red}{S} & \colorbox{green}{H} & \colorbox{green}{H} & \colorbox{red}{S} & \colorbox{red}{S} & \colorbox{red}{S} \\
    \hline
    15 &\colorbox{red}{S} & \colorbox{red}{S} & \colorbox{red}{S} & \colorbox{red}{S} & \colorbox{red}{S} & \colorbox{green}{H} & \colorbox{green}{H} & \colorbox{green}{H} & \colorbox{red}{S} & \colorbox{green}{H} \\
    \hline
    13--14 & \colorbox{red}{S} & \colorbox{red}{S} & \colorbox{red}{S} & \colorbox{red}{S} & \colorbox{red}{S} & \colorbox{green}{H} & \colorbox{green}{H} & \colorbox{green}{H} & \colorbox{green}{H} & \colorbox{green}{H} \\
    \hline
    12 & \colorbox{green}{H} & \colorbox{green}{H} & \colorbox{red}{S} & \colorbox{red}{S} & \colorbox{red}{S} & \colorbox{green}{H} & \colorbox{green}{H} & \colorbox{green}{H} & \colorbox{green}{H} & \colorbox{green}{H} \\
    \hline
    11 & \colorbox{green}{H} & \colorbox{green}{H} & \colorbox{green}{H} & \colorbox{green}{H} & \colorbox{green}{H} & \colorbox{green}{H} & \colorbox{green}{H} & \colorbox{green}{H} & \colorbox{green}{H} & \colorbox{green}{H} \\
    \hline
    10 & \colorbox{green}{H} & \colorbox{green}{H} & \colorbox{green}{H} & \colorbox{green}{H} & \colorbox{green}{H} & \colorbox{green}{H} & \colorbox{green}{H} & \colorbox{green}{H} & \colorbox{green}{H} & \colorbox{green}{H} \\
    \hline
    9 & \colorbox{green}{H} & \colorbox{green}{H} & \colorbox{green}{H} & \colorbox{green}{H} & \colorbox{green}{H} & \colorbox{green}{H} & \colorbox{green}{H} & \colorbox{green}{H} & \colorbox{green}{H} & \colorbox{green}{H} \\
    \hline
    4--8 & \colorbox{green}{H} & \colorbox{green}{H} & \colorbox{green}{H} & \colorbox{green}{H} & \colorbox{green}{H} & \colorbox{green}{H} & \colorbox{green}{H} & \colorbox{green}{H} & \colorbox{green}{H} & \colorbox{green}{H} \\
    \hline
    % \end{small}
  \end{tabular}
  % \vspace*{1cm}
  \begin{tabular}{|l|c|c|c|c|c|c|c|c|c|c|}
    \hline
    \multicolumn{1}{ |l| }{Player hand } & \multicolumn{10}{ |c| }{Soft totals}\\
    \hline
    8--9 & \colorbox{red}{S} & \colorbox{red}{S} & \colorbox{red}{S} & \colorbox{red}{S} & \colorbox{red}{S} & \colorbox{red}{S} & \colorbox{red}{S} & \colorbox{red}{S} & \colorbox{red}{S} & \colorbox{red}{S} \\
    \hline
    7 & \colorbox{red}{S} & \colorbox{red}{S} & \colorbox{red}{S} & \colorbox{red}{S} & \colorbox{red}{S} & \colorbox{red}{S} & \colorbox{red}{S} & \colorbox{green}{H} & \colorbox{green}{H} & \colorbox{green}{H} \\
    \hline
    6 & \colorbox{green}{H} & \colorbox{green}{H} & \colorbox{green}{H} & \colorbox{green}{H} & \colorbox{green}{H} & \colorbox{green}{H} & \colorbox{green}{H} & \colorbox{green}{H} & \colorbox{green}{H} & \colorbox{green}{H} \\
    \hline
    4--5 & \colorbox{green}{H} & \colorbox{green}{H} & \colorbox{green}{H} & \colorbox{green}{H} & \colorbox{green}{H} & \colorbox{green}{H} & \colorbox{green}{H} & \colorbox{green}{H} & \colorbox{green}{H} & \colorbox{green}{H} \\
    \hline
    2--3 & \colorbox{green}{H} & \colorbox{green}{H} & \colorbox{green}{H} & \colorbox{green}{H} & \colorbox{green}{H} & \colorbox{green}{H} & \colorbox{green}{H} & \colorbox{green}{H} & \colorbox{green}{H} & \colorbox{green}{H} \\
    \hline
  \end{tabular}

  \caption{Action table underlying \emph{deterministic table} strategy.\label{tab:det}}
\end{table}


\section*{\texttt{Results}}
AI vs AI
noDecks
Mendacium.


\section*{\texttt{Conclusions}}
Mendacium.

\section*{\texttt{Appendix}---How to run the code}
To run supplied code \texttt{SWI-Prolog}, \texttt{R} and \texttt{R..eal} have to be installed. The adjustments of the game-play can be done by specifying mentioned above parameters which are present at top of the \texttt{main.pl} file.\\
To test, run \texttt{SWIP} in the directory with program and execute: ``\texttt{consult(main).}''~. Then execute: ``\texttt{main.}''~.\\
Depending whether interactive or experimental mode has been chosen users interaction may be required; if one decided to play along the AI modules question whether to hit or stand at each round will be asked.\\

\section*{Appendix A: Playing the game}
To run the script:

 cd ~
 git clone git@github.com:So-Cool/BlackJack.git
 cd BlackJack
 swipl
and then within SWI-Prolog:

consult('main').
main.
After the game is finished there will be a score graph and score table(in the .csv format) produced within the BlackJack directory. The rows in the file are scores from particular round and the columns are in order: dealer, AI1-deterministic table, AI2-deck probabilities, AI3-mixture of previous two, AI4-always hold and finally, if enabled, user's score. The same order applies to vertical bars visible in the plot.

%%%%%%%%%%%%%%%%%%%%%%%%%%%%%%%%%%%%%%%%%%%%%%%%%%%%%%%%%%%%%%%%%%%%%%%%%%%%%%%%
%%%%%%%%%%%%%%%%%%%%%%%%%%%%%%%%  Unused pieces %%%%%%%%%%%%%%%%%%%%%%%%%%%%%%%%
%% Concept of failure driven loops
%%     EXPLORATION STEP  -  *HIT* take a card
%%     EXPLoitation STEP - *HOLD* 
%% \let\thefootnote\relax\footnote{No number}

% \begin{appendices}
    % \section{aapendix a}
% \end{appendices}

% \usepackage{hyperref}
% \usepackage{graphicx}
% \usepackage{subfigure}
% \usepackage{indentfirst} % indent frst paragraph of section
% \newcommand{\ts}{\textsuperscript}

% \begin{figure}[htbp]
% \centering
% \includegraphics[width=0.5\textwidth]{figure1.png}
% % \begin{tiny}
% \caption{Discretized maze.\label{fig:maze}}
% % \end{tiny}
% \vspace{0.2cm}
% \end{figure}

% \begin{figure}[htbp]
%   \begin{subfigure}{.49\linewidth}\centering
%     \includegraphics[width=1.1\textwidth]{figure2.png}
%     \caption{Neuron 1\label{fig:FiringPosition_N1}}
%   \end{subfigure}
%   \begin{subfigure}{.49\linewidth}\centering
%     \includegraphics[width=1.1\textwidth]{figure3.png}
%     \caption{Neuron 2\label{fig:FiringPosition_N2}}
%   \end{subfigure}\\[1ex]

%     \begin{subfigure}{.49\linewidth}\centering
%     \includegraphics[width=1.1\textwidth]{figure3.png}
%     \caption{Neuron 3\label{fig:FiringPosition_N3}}
%   \end{subfigure}
%   \begin{subfigure}{.49\linewidth}\centering
%     \includegraphics[width=1.1\textwidth]{figure4.png}
%     \caption{Neuron 4\label{fig:FiringPosition_N4}}
%   \end{subfigure}

%   \caption{Neurons firing rates based on position change of a rat.\label{fig:FiringPosition_ALL}}
% \end{figure}

% \begin{figure}[h]
%   \begin{subfigure}{.99\linewidth}\centering
%     \includegraphics[width=1.1\textwidth]{figure6_a.png}
%     \caption{\label{fig:3d_n1_a}}
%   \end{subfigure}\\[1ex]
%   \begin{subfigure}{.99\linewidth}\centering
%     \includegraphics[width=1.1\textwidth]{figure6_b.png}
%     \caption{\label{fig:3d_n1_b}}
%   \end{subfigure}

%   \caption{Neuron 1---spatial activity.\label{fig:3d_n1}}
% \end{figure}


%%%%%%%%%%%%%%%%%%%%%%%%%%%%%%%%%%%%%%%%%%%%%%%%%%%%%%%%%%%%%%%%%%%%%%%%%%%%%%%%
\end{document}

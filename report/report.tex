% Report --- ks1591
\documentclass[12pt,a4paper,twocolumn]{article}
\usepackage{times} % times font
\usepackage{mathptmx} % times font in maths
\usepackage{fullpage}
\usepackage[top=0.7in, bottom=0.7in, left=0.7in, right=0.7in]{geometry}
\usepackage{multirow} %in tables
\usepackage{caption} % in tables
\pagenumbering{gobble}
\newcommand{\HRule}{\rule{\linewidth}{0.5mm}}

\usepackage[pdftex]{graphicx}
\usepackage{lipsum}
\usepackage{amsmath}

% for multi figures
\usepackage{graphicx}
\usepackage{caption}
\usepackage{subcaption}
\usepackage{float}

\usepackage[usenames,dvipsnames]{color}

% \usepackage{hyperref}
% \usepackage{graphicx}
% \usepackage{subfigure}
% \usepackage{indentfirst} % indent frst paragraph of section

% \usepackage[usenames,dvipsnames]{color}

% \newcommand{\ts}{\textsuperscript}

\begin{document}

\twocolumn[
\begin{@twocolumnfalse}
\begin{center}
	\begin{large}
	{\HRule \\[0.2cm]}
	\textsc{Artificial intelligence algorithms for BlackJack}
	{\HRule \\[0.3cm]}
	\end{large}

	\begin{minipage}{ 0.44\textwidth }
		\begin{flushleft}
			Kacper \textbf{Sokol}\\
			\texttt{ks1591} --- 3GGK1
		\end{flushleft}
	\end{minipage}
	\begin{minipage}{ 0.44\textwidth }
		\begin{flushright}
			{COMS30106 $|$ Artificial Intelligence \&\\Logic Programming\\
			Assignment 2: BlackJack\\[0.3cm]}
		\end{flushright}
	\end{minipage}
\end{center}
\end{@twocolumnfalse}
] % \lipsum[1]~\\[0.4cm]

\begin{abstract}
This paper reviews most common strategies used by \emph{BlackJack}\footnote{For more details see:\\http://en.wikipedia.org/wiki/Blackjack~.} players to decrease \emph{house edge}\footnote{A statistical advantage of the house(casino) over player.}. Algorithms such as: \emph{deterministic table}, \emph{deck probabilities}, \emph{stand on all} and combination of first two are tested.\\
For this purpose BlackJack engine is implemented in \texttt{SWI-Prolog} programming language and \texttt{R} statistical package, where the later is accessed via \texttt{R..eal}\footnote{For more details see:\\http://www.swi-prolog.org/pack/list?p=real~.} interface.\\
Finally, results of the experiment are presented and examined.\\

\begin{small}
\noindent
The source code together with this report is available at:\\https://github.com/So-Cool/BlackJack~.
\end{small}
\end{abstract}

\section*{\texttt{The experiment}}
To examine aforementioned strategies BlackJack game was implemented with number of parameters being user configurable. There are two house strategies implemented, which are documented to be the most popular: \emph{H17} and \emph{S17}. i.e.\ dealer can... only one card is revealed to the players. Furthermore, the house plays as last in each round therefore being aware of the table situation.\\

At each stage the player can either:  *HIT* or *HOLD*\\

use repl to make statistical computations\\

code description: key choices; game theory and results for different number of decks for different strategies\\

\section*{\texttt{Strategies}}

\begin{description}
\item{\textbf{Deterministic table}} \hfill\\
Rat begin in TL or BL and run toward left junction. The movable barrier is placed there to direct the animal into the central arm.
\item{\textbf{Deck probabilities}} \hfill\\
The rat gets to the right junction---decision point---where it choses left or right turn.
\item{\textbf{Stand on all}} \hfill\\
Only the choice which leads to the point at the same height as the starting position is awarded.
\item{\textbf{Probabilistic choice}} \hfill\\
After reaching TR or BR the rat returns to the right junction where another movable barrier is placed to direct the rat into central arm.
\end{description}

% \colorbox{green}{H}
\begin{table}[t]
  \tiny
  \begin{tabular}{|l|c|c|c|c|c|c|c|c|c|c|}
    \hline
    \multicolumn{1}{ |l| }{\multirow{3}{*}{Player hand} } & \multicolumn{10}{ |c| }{Dealer's face-up card}\\
    \cline{2-11}
     & \multicolumn{10}{ |c| }{Hard totals}\\
    \cline{2-11}
     & 2 & 3 & 4 & 5 & 6 & 7 & 8 & 9 & 10 & A \\
    \hline
    17--20 & \colorbox{red}{S} & \colorbox{red}{S} & \colorbox{red}{S} & \colorbox{red}{S} & \colorbox{red}{S} & \colorbox{red}{S} & \colorbox{red}{S} & \colorbox{red}{S} & \colorbox{red}{S} & \colorbox{red}{S} \\
    \hline
    16 & \colorbox{red}{S} & \colorbox{red}{S} & \colorbox{red}{S} & \colorbox{red}{S} & \colorbox{red}{S} & \colorbox{green}{H} & \colorbox{green}{H} & \colorbox{red}{S} & \colorbox{red}{S} & \colorbox{red}{S} \\
    \hline
    15 &\colorbox{red}{S} & \colorbox{red}{S} & \colorbox{red}{S} & \colorbox{red}{S} & \colorbox{red}{S} & \colorbox{green}{H} & \colorbox{green}{H} & \colorbox{green}{H} & \colorbox{red}{S} & \colorbox{green}{H} \\
    \hline
    13--14 & \colorbox{red}{S} & \colorbox{red}{S} & \colorbox{red}{S} & \colorbox{red}{S} & \colorbox{red}{S} & \colorbox{green}{H} & \colorbox{green}{H} & \colorbox{green}{H} & \colorbox{green}{H} & \colorbox{green}{H} \\
    \hline
    12 & \colorbox{green}{H} & \colorbox{green}{H} & \colorbox{red}{S} & \colorbox{red}{S} & \colorbox{red}{S} & \colorbox{green}{H} & \colorbox{green}{H} & \colorbox{green}{H} & \colorbox{green}{H} & \colorbox{green}{H} \\
    \hline
    11 & \colorbox{green}{H} & \colorbox{green}{H} & \colorbox{green}{H} & \colorbox{green}{H} & \colorbox{green}{H} & \colorbox{green}{H} & \colorbox{green}{H} & \colorbox{green}{H} & \colorbox{green}{H} & \colorbox{green}{H} \\
    \hline
    10 & \colorbox{green}{H} & \colorbox{green}{H} & \colorbox{green}{H} & \colorbox{green}{H} & \colorbox{green}{H} & \colorbox{green}{H} & \colorbox{green}{H} & \colorbox{green}{H} & \colorbox{green}{H} & \colorbox{green}{H} \\
    \hline
    9 & \colorbox{green}{H} & \colorbox{green}{H} & \colorbox{green}{H} & \colorbox{green}{H} & \colorbox{green}{H} & \colorbox{green}{H} & \colorbox{green}{H} & \colorbox{green}{H} & \colorbox{green}{H} & \colorbox{green}{H} \\
    \hline
    4--8 & \colorbox{green}{H} & \colorbox{green}{H} & \colorbox{green}{H} & \colorbox{green}{H} & \colorbox{green}{H} & \colorbox{green}{H} & \colorbox{green}{H} & \colorbox{green}{H} & \colorbox{green}{H} & \colorbox{green}{H} \\
    \hline
    % \end{small}
  \end{tabular}
  % \vspace*{1cm}
  \begin{tabular}{|l|c|c|c|c|c|c|c|c|c|c|}
    \hline
    \multicolumn{1}{ |l| }{Player hand } & \multicolumn{10}{ |c| }{Soft totals}\\
    \hline
    8--9 & \colorbox{red}{S} & \colorbox{red}{S} & \colorbox{red}{S} & \colorbox{red}{S} & \colorbox{red}{S} & \colorbox{red}{S} & \colorbox{red}{S} & \colorbox{red}{S} & \colorbox{red}{S} & \colorbox{red}{S} \\
    \hline
    7 & \colorbox{red}{S} & \colorbox{red}{S} & \colorbox{red}{S} & \colorbox{red}{S} & \colorbox{red}{S} & \colorbox{red}{S} & \colorbox{red}{S} & \colorbox{green}{H} & \colorbox{green}{H} & \colorbox{green}{H} \\
    \hline
    6 & \colorbox{green}{H} & \colorbox{green}{H} & \colorbox{green}{H} & \colorbox{green}{H} & \colorbox{green}{H} & \colorbox{green}{H} & \colorbox{green}{H} & \colorbox{green}{H} & \colorbox{green}{H} & \colorbox{green}{H} \\
    \hline
    4--5 & \colorbox{green}{H} & \colorbox{green}{H} & \colorbox{green}{H} & \colorbox{green}{H} & \colorbox{green}{H} & \colorbox{green}{H} & \colorbox{green}{H} & \colorbox{green}{H} & \colorbox{green}{H} & \colorbox{green}{H} \\
    \hline
    2--3 & \colorbox{green}{H} & \colorbox{green}{H} & \colorbox{green}{H} & \colorbox{green}{H} & \colorbox{green}{H} & \colorbox{green}{H} & \colorbox{green}{H} & \colorbox{green}{H} & \colorbox{green}{H} & \colorbox{green}{H} \\
    \hline
  \end{tabular}

  \caption{Action table underlying \emph{deterministic table} strategy. .\label{tab:det}}
\end{table}


It would be nice to mention that split is not imlemented and why and that it would be even incer o do so

sovf vs. hard dealer
AI vs AI
noDecks

Player1 - deterministic table
Player2 - calculates deck probabilities
Player3 - combines strategies of Player1\&2
Player4 - never takes a card
Player5 - my przeczucie

% \let\thefootnote\relax\footnote{Theta Rhythms Coordinate\\Hippocampal–--Prefrontal Interactions\\in a Spatial Memory Task\\Matthew W. Jones, Matthew A. Wilson}

% \begin{figure}[htbp]
% \centering
% \includegraphics[width=0.5\textwidth]{figure1.png}
% % \begin{tiny}
% \caption{Discretized maze.\label{fig:maze}}
% % \end{tiny}
% \vspace{0.2cm}
% \end{figure}

\section*{\texttt{Decision making}}

% \begin{figure}[htbp]
%   \begin{subfigure}{.49\linewidth}\centering
%     \includegraphics[width=1.1\textwidth]{figure2.png}
%     \caption{Neuron 1\label{fig:FiringPosition_N1}}
%   \end{subfigure}
%   \begin{subfigure}{.49\linewidth}\centering
%     \includegraphics[width=1.1\textwidth]{figure3.png}
%     \caption{Neuron 2\label{fig:FiringPosition_N2}}
%   \end{subfigure}\\[1ex]

%     \begin{subfigure}{.49\linewidth}\centering
%     \includegraphics[width=1.1\textwidth]{figure3.png}
%     \caption{Neuron 3\label{fig:FiringPosition_N3}}
%   \end{subfigure}
%   \begin{subfigure}{.49\linewidth}\centering
%     \includegraphics[width=1.1\textwidth]{figure4.png}
%     \caption{Neuron 4\label{fig:FiringPosition_N4}}
%   \end{subfigure}

%   \caption{Neurons firing rates based on position change of a rat.\label{fig:FiringPosition_ALL}}
% \end{figure}

\section*{\texttt{Analysis}}

% \begin{figure}[h]
%   \begin{subfigure}{.99\linewidth}\centering
%     \includegraphics[width=1.1\textwidth]{figure6_a.png}
%     \caption{\label{fig:3d_n1_a}}
%   \end{subfigure}\\[1ex]
%   \begin{subfigure}{.99\linewidth}\centering
%     \includegraphics[width=1.1\textwidth]{figure6_b.png}
%     \caption{\label{fig:3d_n1_b}}
%   \end{subfigure}

%   \caption{Neuron 1---spatial activity.\label{fig:3d_n1}}
% \end{figure}


\section*{\texttt{Conclusions}}


%%%%%%%%%%%%%%%%%%%%%%%%%%%%%%%%%%%%%%%%%%%%%%%%%%%%%%%%%%%%%%%%%%%%%%%%%%%%%%%%
%%%%%%%%%%%%%%%%%%%%%%%%%%%%%%%%  Unused pieces %%%%%%%%%%%%%%%%%%%%%%%%%%%%%%%%
%% Concept of failure driven loops
%%     EXPLORATION STEP  -  *HIT* take a card
%%     EXPLoitation STEP - *HOLD* 
%% back here
%%%%%%%%%%%%%%%%%%%%%%%%%%%%%%%%%%%%%%%%%%%%%%%%%%%%%%%%%%%%%%%%%%%%%%%%%%%%%%%%
\end{document}
